\section{Time Series}

A time series, also known as discrete time signal, is a sequence of observations taken periodically in time. We can use time series to perform many tasks such as predictions of future values, behaviour analysis or information extraction. Examples of time series are audio signals, industrial instrument measures or diary finantial activity.

A system can be determined comparing the input and the output. We call the system a filter if it is linear and time invariant. Considering the dynamic system as a black box, we can estimate the transference function or the impulse response to taht filter.

We can also consider \textbf{multivariate} time series, where some values of the time series have an influence on the other values in different or the same time instant. We can \textbf{classify} the time series in two wide types:

\begin{itemize}
  \item Determinist: based in dynamic systems, they exploit the phisics of the generation algorithm of the time series.
        \item Stochastic: where the series are realizations of a stochastic process, which can be modelated.
\end{itemize}

In this subject, we will focus on stochastic models.
