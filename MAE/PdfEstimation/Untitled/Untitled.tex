\documentclass[11pt]{article}

    \usepackage[breakable]{tcolorbox}
    \usepackage{parskip} % Stop auto-indenting (to mimic markdown behaviour)
    
    \usepackage{iftex}
    \ifPDFTeX
    	\usepackage[T1]{fontenc}
    	\usepackage{mathpazo}
    \else
    	\usepackage{fontspec}
    \fi

    % Basic figure setup, for now with no caption control since it's done
    % automatically by Pandoc (which extracts ![](path) syntax from Markdown).
    \usepackage{graphicx}
    % Maintain compatibility with old templates. Remove in nbconvert 6.0
    \let\Oldincludegraphics\includegraphics
    % Ensure that by default, figures have no caption (until we provide a
    % proper Figure object with a Caption API and a way to capture that
    % in the conversion process - todo).
    \usepackage{caption}
    \DeclareCaptionFormat{nocaption}{}
    \captionsetup{format=nocaption,aboveskip=0pt,belowskip=0pt}

    \usepackage{float}
    \floatplacement{figure}{H} % forces figures to be placed at the correct location
    \usepackage{xcolor} % Allow colors to be defined
    \usepackage{enumerate} % Needed for markdown enumerations to work
    \usepackage{geometry} % Used to adjust the document margins
    \usepackage{amsmath} % Equations
    \usepackage{amssymb} % Equations
    \usepackage{textcomp} % defines textquotesingle
    % Hack from http://tex.stackexchange.com/a/47451/13684:
    \AtBeginDocument{%
        \def\PYZsq{\textquotesingle}% Upright quotes in Pygmentized code
    }
    \usepackage{upquote} % Upright quotes for verbatim code
    \usepackage{eurosym} % defines \euro
    \usepackage[mathletters]{ucs} % Extended unicode (utf-8) support
    \usepackage{fancyvrb} % verbatim replacement that allows latex
    \usepackage{grffile} % extends the file name processing of package graphics 
                         % to support a larger range
    \makeatletter % fix for old versions of grffile with XeLaTeX
    \@ifpackagelater{grffile}{2019/11/01}
    {
      % Do nothing on new versions
    }
    {
      \def\Gread@@xetex#1{%
        \IfFileExists{"\Gin@base".bb}%
        {\Gread@eps{\Gin@base.bb}}%
        {\Gread@@xetex@aux#1}%
      }
    }
    \makeatother
    \usepackage[Export]{adjustbox} % Used to constrain images to a maximum size
    \adjustboxset{max size={0.9\linewidth}{0.9\paperheight}}

    % The hyperref package gives us a pdf with properly built
    % internal navigation ('pdf bookmarks' for the table of contents,
    % internal cross-reference links, web links for URLs, etc.)
    \usepackage{hyperref}
    % The default LaTeX title has an obnoxious amount of whitespace. By default,
    % titling removes some of it. It also provides customization options.
    \usepackage{titling}
    \usepackage{longtable} % longtable support required by pandoc >1.10
    \usepackage{booktabs}  % table support for pandoc > 1.12.2
    \usepackage[inline]{enumitem} % IRkernel/repr support (it uses the enumerate* environment)
    \usepackage[normalem]{ulem} % ulem is needed to support strikethroughs (\sout)
                                % normalem makes italics be italics, not underlines
    \usepackage{mathrsfs}
    

    
    % Colors for the hyperref package
    \definecolor{urlcolor}{rgb}{0,.145,.698}
    \definecolor{linkcolor}{rgb}{.71,0.21,0.01}
    \definecolor{citecolor}{rgb}{.12,.54,.11}

    % ANSI colors
    \definecolor{ansi-black}{HTML}{3E424D}
    \definecolor{ansi-black-intense}{HTML}{282C36}
    \definecolor{ansi-red}{HTML}{E75C58}
    \definecolor{ansi-red-intense}{HTML}{B22B31}
    \definecolor{ansi-green}{HTML}{00A250}
    \definecolor{ansi-green-intense}{HTML}{007427}
    \definecolor{ansi-yellow}{HTML}{DDB62B}
    \definecolor{ansi-yellow-intense}{HTML}{B27D12}
    \definecolor{ansi-blue}{HTML}{208FFB}
    \definecolor{ansi-blue-intense}{HTML}{0065CA}
    \definecolor{ansi-magenta}{HTML}{D160C4}
    \definecolor{ansi-magenta-intense}{HTML}{A03196}
    \definecolor{ansi-cyan}{HTML}{60C6C8}
    \definecolor{ansi-cyan-intense}{HTML}{258F8F}
    \definecolor{ansi-white}{HTML}{C5C1B4}
    \definecolor{ansi-white-intense}{HTML}{A1A6B2}
    \definecolor{ansi-default-inverse-fg}{HTML}{FFFFFF}
    \definecolor{ansi-default-inverse-bg}{HTML}{000000}

    % common color for the border for error outputs.
    \definecolor{outerrorbackground}{HTML}{FFDFDF}

    % commands and environments needed by pandoc snippets
    % extracted from the output of `pandoc -s`
    \providecommand{\tightlist}{%
      \setlength{\itemsep}{0pt}\setlength{\parskip}{0pt}}
    \DefineVerbatimEnvironment{Highlighting}{Verbatim}{commandchars=\\\{\}}
    % Add ',fontsize=\small' for more characters per line
    \newenvironment{Shaded}{}{}
    \newcommand{\KeywordTok}[1]{\textcolor[rgb]{0.00,0.44,0.13}{\textbf{{#1}}}}
    \newcommand{\DataTypeTok}[1]{\textcolor[rgb]{0.56,0.13,0.00}{{#1}}}
    \newcommand{\DecValTok}[1]{\textcolor[rgb]{0.25,0.63,0.44}{{#1}}}
    \newcommand{\BaseNTok}[1]{\textcolor[rgb]{0.25,0.63,0.44}{{#1}}}
    \newcommand{\FloatTok}[1]{\textcolor[rgb]{0.25,0.63,0.44}{{#1}}}
    \newcommand{\CharTok}[1]{\textcolor[rgb]{0.25,0.44,0.63}{{#1}}}
    \newcommand{\StringTok}[1]{\textcolor[rgb]{0.25,0.44,0.63}{{#1}}}
    \newcommand{\CommentTok}[1]{\textcolor[rgb]{0.38,0.63,0.69}{\textit{{#1}}}}
    \newcommand{\OtherTok}[1]{\textcolor[rgb]{0.00,0.44,0.13}{{#1}}}
    \newcommand{\AlertTok}[1]{\textcolor[rgb]{1.00,0.00,0.00}{\textbf{{#1}}}}
    \newcommand{\FunctionTok}[1]{\textcolor[rgb]{0.02,0.16,0.49}{{#1}}}
    \newcommand{\RegionMarkerTok}[1]{{#1}}
    \newcommand{\ErrorTok}[1]{\textcolor[rgb]{1.00,0.00,0.00}{\textbf{{#1}}}}
    \newcommand{\NormalTok}[1]{{#1}}
    
    % Additional commands for more recent versions of Pandoc
    \newcommand{\ConstantTok}[1]{\textcolor[rgb]{0.53,0.00,0.00}{{#1}}}
    \newcommand{\SpecialCharTok}[1]{\textcolor[rgb]{0.25,0.44,0.63}{{#1}}}
    \newcommand{\VerbatimStringTok}[1]{\textcolor[rgb]{0.25,0.44,0.63}{{#1}}}
    \newcommand{\SpecialStringTok}[1]{\textcolor[rgb]{0.73,0.40,0.53}{{#1}}}
    \newcommand{\ImportTok}[1]{{#1}}
    \newcommand{\DocumentationTok}[1]{\textcolor[rgb]{0.73,0.13,0.13}{\textit{{#1}}}}
    \newcommand{\AnnotationTok}[1]{\textcolor[rgb]{0.38,0.63,0.69}{\textbf{\textit{{#1}}}}}
    \newcommand{\CommentVarTok}[1]{\textcolor[rgb]{0.38,0.63,0.69}{\textbf{\textit{{#1}}}}}
    \newcommand{\VariableTok}[1]{\textcolor[rgb]{0.10,0.09,0.49}{{#1}}}
    \newcommand{\ControlFlowTok}[1]{\textcolor[rgb]{0.00,0.44,0.13}{\textbf{{#1}}}}
    \newcommand{\OperatorTok}[1]{\textcolor[rgb]{0.40,0.40,0.40}{{#1}}}
    \newcommand{\BuiltInTok}[1]{{#1}}
    \newcommand{\ExtensionTok}[1]{{#1}}
    \newcommand{\PreprocessorTok}[1]{\textcolor[rgb]{0.74,0.48,0.00}{{#1}}}
    \newcommand{\AttributeTok}[1]{\textcolor[rgb]{0.49,0.56,0.16}{{#1}}}
    \newcommand{\InformationTok}[1]{\textcolor[rgb]{0.38,0.63,0.69}{\textbf{\textit{{#1}}}}}
    \newcommand{\WarningTok}[1]{\textcolor[rgb]{0.38,0.63,0.69}{\textbf{\textit{{#1}}}}}
    
    
    % Define a nice break command that doesn't care if a line doesn't already
    % exist.
    \def\br{\hspace*{\fill} \\* }
    % Math Jax compatibility definitions
    \def\gt{>}
    \def\lt{<}
    \let\Oldtex\TeX
    \let\Oldlatex\LaTeX
    \renewcommand{\TeX}{\textrm{\Oldtex}}
    \renewcommand{\LaTeX}{\textrm{\Oldlatex}}
    % Document parameters
    % Document title
    \title{Ejercicios estimación }

    
\usepackage{mdframed}


\newenvironment{problem}[2][Problema]
    { \begin{mdframed}[backgroundcolor=gray!20] \textbf{#1 #2} \\}
    {  \end{mdframed}}

% Define solution environment
\newenvironment{solution}
    {\textit{Solución}}
    {}
    

    \usepackage{fancyhdr} % Headers and footers
    \pagestyle{fancy} % All pages have headers and footers
    \fancyhead{}\renewcommand{\headrulewidth}{0pt} % Blank out the default header
    \fancyfoot[L]{} % Custom footer text
    \fancyfoot[C]{} % Custom footer text
    \fancyfoot[R]{\thepage} % Custom footer text
    \newcommand{\note}[1]{\marginpar{\scriptsize \textcolor{red}{#1}}} % Enables comments in red on margin
    
    
    
    
% Pygments definitions
\makeatletter
\def\PY@reset{\let\PY@it=\relax \let\PY@bf=\relax%
    \let\PY@ul=\relax \let\PY@tc=\relax%
    \let\PY@bc=\relax \let\PY@ff=\relax}
\def\PY@tok#1{\csname PY@tok@#1\endcsname}
\def\PY@toks#1+{\ifx\relax#1\empty\else%
    \PY@tok{#1}\expandafter\PY@toks\fi}
\def\PY@do#1{\PY@bc{\PY@tc{\PY@ul{%
    \PY@it{\PY@bf{\PY@ff{#1}}}}}}}
\def\PY#1#2{\PY@reset\PY@toks#1+\relax+\PY@do{#2}}

\expandafter\def\csname PY@tok@w\endcsname{\def\PY@tc##1{\textcolor[rgb]{0.73,0.73,0.73}{##1}}}
\expandafter\def\csname PY@tok@c\endcsname{\let\PY@it=\textit\def\PY@tc##1{\textcolor[rgb]{0.25,0.50,0.50}{##1}}}
\expandafter\def\csname PY@tok@cp\endcsname{\def\PY@tc##1{\textcolor[rgb]{0.74,0.48,0.00}{##1}}}
\expandafter\def\csname PY@tok@k\endcsname{\let\PY@bf=\textbf\def\PY@tc##1{\textcolor[rgb]{0.00,0.50,0.00}{##1}}}
\expandafter\def\csname PY@tok@kp\endcsname{\def\PY@tc##1{\textcolor[rgb]{0.00,0.50,0.00}{##1}}}
\expandafter\def\csname PY@tok@kt\endcsname{\def\PY@tc##1{\textcolor[rgb]{0.69,0.00,0.25}{##1}}}
\expandafter\def\csname PY@tok@o\endcsname{\def\PY@tc##1{\textcolor[rgb]{0.40,0.40,0.40}{##1}}}
\expandafter\def\csname PY@tok@ow\endcsname{\let\PY@bf=\textbf\def\PY@tc##1{\textcolor[rgb]{0.67,0.13,1.00}{##1}}}
\expandafter\def\csname PY@tok@nb\endcsname{\def\PY@tc##1{\textcolor[rgb]{0.00,0.50,0.00}{##1}}}
\expandafter\def\csname PY@tok@nf\endcsname{\def\PY@tc##1{\textcolor[rgb]{0.00,0.00,1.00}{##1}}}
\expandafter\def\csname PY@tok@nc\endcsname{\let\PY@bf=\textbf\def\PY@tc##1{\textcolor[rgb]{0.00,0.00,1.00}{##1}}}
\expandafter\def\csname PY@tok@nn\endcsname{\let\PY@bf=\textbf\def\PY@tc##1{\textcolor[rgb]{0.00,0.00,1.00}{##1}}}
\expandafter\def\csname PY@tok@ne\endcsname{\let\PY@bf=\textbf\def\PY@tc##1{\textcolor[rgb]{0.82,0.25,0.23}{##1}}}
\expandafter\def\csname PY@tok@nv\endcsname{\def\PY@tc##1{\textcolor[rgb]{0.10,0.09,0.49}{##1}}}
\expandafter\def\csname PY@tok@no\endcsname{\def\PY@tc##1{\textcolor[rgb]{0.53,0.00,0.00}{##1}}}
\expandafter\def\csname PY@tok@nl\endcsname{\def\PY@tc##1{\textcolor[rgb]{0.63,0.63,0.00}{##1}}}
\expandafter\def\csname PY@tok@ni\endcsname{\let\PY@bf=\textbf\def\PY@tc##1{\textcolor[rgb]{0.60,0.60,0.60}{##1}}}
\expandafter\def\csname PY@tok@na\endcsname{\def\PY@tc##1{\textcolor[rgb]{0.49,0.56,0.16}{##1}}}
\expandafter\def\csname PY@tok@nt\endcsname{\let\PY@bf=\textbf\def\PY@tc##1{\textcolor[rgb]{0.00,0.50,0.00}{##1}}}
\expandafter\def\csname PY@tok@nd\endcsname{\def\PY@tc##1{\textcolor[rgb]{0.67,0.13,1.00}{##1}}}
\expandafter\def\csname PY@tok@s\endcsname{\def\PY@tc##1{\textcolor[rgb]{0.73,0.13,0.13}{##1}}}
\expandafter\def\csname PY@tok@sd\endcsname{\let\PY@it=\textit\def\PY@tc##1{\textcolor[rgb]{0.73,0.13,0.13}{##1}}}
\expandafter\def\csname PY@tok@si\endcsname{\let\PY@bf=\textbf\def\PY@tc##1{\textcolor[rgb]{0.73,0.40,0.53}{##1}}}
\expandafter\def\csname PY@tok@se\endcsname{\let\PY@bf=\textbf\def\PY@tc##1{\textcolor[rgb]{0.73,0.40,0.13}{##1}}}
\expandafter\def\csname PY@tok@sr\endcsname{\def\PY@tc##1{\textcolor[rgb]{0.73,0.40,0.53}{##1}}}
\expandafter\def\csname PY@tok@ss\endcsname{\def\PY@tc##1{\textcolor[rgb]{0.10,0.09,0.49}{##1}}}
\expandafter\def\csname PY@tok@sx\endcsname{\def\PY@tc##1{\textcolor[rgb]{0.00,0.50,0.00}{##1}}}
\expandafter\def\csname PY@tok@m\endcsname{\def\PY@tc##1{\textcolor[rgb]{0.40,0.40,0.40}{##1}}}
\expandafter\def\csname PY@tok@gh\endcsname{\let\PY@bf=\textbf\def\PY@tc##1{\textcolor[rgb]{0.00,0.00,0.50}{##1}}}
\expandafter\def\csname PY@tok@gu\endcsname{\let\PY@bf=\textbf\def\PY@tc##1{\textcolor[rgb]{0.50,0.00,0.50}{##1}}}
\expandafter\def\csname PY@tok@gd\endcsname{\def\PY@tc##1{\textcolor[rgb]{0.63,0.00,0.00}{##1}}}
\expandafter\def\csname PY@tok@gi\endcsname{\def\PY@tc##1{\textcolor[rgb]{0.00,0.63,0.00}{##1}}}
\expandafter\def\csname PY@tok@gr\endcsname{\def\PY@tc##1{\textcolor[rgb]{1.00,0.00,0.00}{##1}}}
\expandafter\def\csname PY@tok@ge\endcsname{\let\PY@it=\textit}
\expandafter\def\csname PY@tok@gs\endcsname{\let\PY@bf=\textbf}
\expandafter\def\csname PY@tok@gp\endcsname{\let\PY@bf=\textbf\def\PY@tc##1{\textcolor[rgb]{0.00,0.00,0.50}{##1}}}
\expandafter\def\csname PY@tok@go\endcsname{\def\PY@tc##1{\textcolor[rgb]{0.53,0.53,0.53}{##1}}}
\expandafter\def\csname PY@tok@gt\endcsname{\def\PY@tc##1{\textcolor[rgb]{0.00,0.27,0.87}{##1}}}
\expandafter\def\csname PY@tok@err\endcsname{\def\PY@bc##1{\setlength{\fboxsep}{0pt}\fcolorbox[rgb]{1.00,0.00,0.00}{1,1,1}{\strut ##1}}}
\expandafter\def\csname PY@tok@kc\endcsname{\let\PY@bf=\textbf\def\PY@tc##1{\textcolor[rgb]{0.00,0.50,0.00}{##1}}}
\expandafter\def\csname PY@tok@kd\endcsname{\let\PY@bf=\textbf\def\PY@tc##1{\textcolor[rgb]{0.00,0.50,0.00}{##1}}}
\expandafter\def\csname PY@tok@kn\endcsname{\let\PY@bf=\textbf\def\PY@tc##1{\textcolor[rgb]{0.00,0.50,0.00}{##1}}}
\expandafter\def\csname PY@tok@kr\endcsname{\let\PY@bf=\textbf\def\PY@tc##1{\textcolor[rgb]{0.00,0.50,0.00}{##1}}}
\expandafter\def\csname PY@tok@bp\endcsname{\def\PY@tc##1{\textcolor[rgb]{0.00,0.50,0.00}{##1}}}
\expandafter\def\csname PY@tok@fm\endcsname{\def\PY@tc##1{\textcolor[rgb]{0.00,0.00,1.00}{##1}}}
\expandafter\def\csname PY@tok@vc\endcsname{\def\PY@tc##1{\textcolor[rgb]{0.10,0.09,0.49}{##1}}}
\expandafter\def\csname PY@tok@vg\endcsname{\def\PY@tc##1{\textcolor[rgb]{0.10,0.09,0.49}{##1}}}
\expandafter\def\csname PY@tok@vi\endcsname{\def\PY@tc##1{\textcolor[rgb]{0.10,0.09,0.49}{##1}}}
\expandafter\def\csname PY@tok@vm\endcsname{\def\PY@tc##1{\textcolor[rgb]{0.10,0.09,0.49}{##1}}}
\expandafter\def\csname PY@tok@sa\endcsname{\def\PY@tc##1{\textcolor[rgb]{0.73,0.13,0.13}{##1}}}
\expandafter\def\csname PY@tok@sb\endcsname{\def\PY@tc##1{\textcolor[rgb]{0.73,0.13,0.13}{##1}}}
\expandafter\def\csname PY@tok@sc\endcsname{\def\PY@tc##1{\textcolor[rgb]{0.73,0.13,0.13}{##1}}}
\expandafter\def\csname PY@tok@dl\endcsname{\def\PY@tc##1{\textcolor[rgb]{0.73,0.13,0.13}{##1}}}
\expandafter\def\csname PY@tok@s2\endcsname{\def\PY@tc##1{\textcolor[rgb]{0.73,0.13,0.13}{##1}}}
\expandafter\def\csname PY@tok@sh\endcsname{\def\PY@tc##1{\textcolor[rgb]{0.73,0.13,0.13}{##1}}}
\expandafter\def\csname PY@tok@s1\endcsname{\def\PY@tc##1{\textcolor[rgb]{0.73,0.13,0.13}{##1}}}
\expandafter\def\csname PY@tok@mb\endcsname{\def\PY@tc##1{\textcolor[rgb]{0.40,0.40,0.40}{##1}}}
\expandafter\def\csname PY@tok@mf\endcsname{\def\PY@tc##1{\textcolor[rgb]{0.40,0.40,0.40}{##1}}}
\expandafter\def\csname PY@tok@mh\endcsname{\def\PY@tc##1{\textcolor[rgb]{0.40,0.40,0.40}{##1}}}
\expandafter\def\csname PY@tok@mi\endcsname{\def\PY@tc##1{\textcolor[rgb]{0.40,0.40,0.40}{##1}}}
\expandafter\def\csname PY@tok@il\endcsname{\def\PY@tc##1{\textcolor[rgb]{0.40,0.40,0.40}{##1}}}
\expandafter\def\csname PY@tok@mo\endcsname{\def\PY@tc##1{\textcolor[rgb]{0.40,0.40,0.40}{##1}}}
\expandafter\def\csname PY@tok@ch\endcsname{\let\PY@it=\textit\def\PY@tc##1{\textcolor[rgb]{0.25,0.50,0.50}{##1}}}
\expandafter\def\csname PY@tok@cm\endcsname{\let\PY@it=\textit\def\PY@tc##1{\textcolor[rgb]{0.25,0.50,0.50}{##1}}}
\expandafter\def\csname PY@tok@cpf\endcsname{\let\PY@it=\textit\def\PY@tc##1{\textcolor[rgb]{0.25,0.50,0.50}{##1}}}
\expandafter\def\csname PY@tok@c1\endcsname{\let\PY@it=\textit\def\PY@tc##1{\textcolor[rgb]{0.25,0.50,0.50}{##1}}}
\expandafter\def\csname PY@tok@cs\endcsname{\let\PY@it=\textit\def\PY@tc##1{\textcolor[rgb]{0.25,0.50,0.50}{##1}}}

\def\PYZbs{\char`\\}
\def\PYZus{\char`\_}
\def\PYZob{\char`\{}
\def\PYZcb{\char`\}}
\def\PYZca{\char`\^}
\def\PYZam{\char`\&}
\def\PYZlt{\char`\<}
\def\PYZgt{\char`\>}
\def\PYZsh{\char`\#}
\def\PYZpc{\char`\%}
\def\PYZdl{\char`\$}
\def\PYZhy{\char`\-}
\def\PYZsq{\char`\'}
\def\PYZdq{\char`\"}
\def\PYZti{\char`\~}
% for compatibility with earlier versions
\def\PYZat{@}
\def\PYZlb{[}
\def\PYZrb{]}
\makeatother


    % For linebreaks inside Verbatim environment from package fancyvrb. 
    \makeatletter
        \newbox\Wrappedcontinuationbox 
        \newbox\Wrappedvisiblespacebox 
        \newcommand*\Wrappedvisiblespace {\textcolor{red}{\textvisiblespace}} 
        \newcommand*\Wrappedcontinuationsymbol {\textcolor{red}{\llap{\tiny$\m@th\hookrightarrow$}}} 
        \newcommand*\Wrappedcontinuationindent {3ex } 
        \newcommand*\Wrappedafterbreak {\kern\Wrappedcontinuationindent\copy\Wrappedcontinuationbox} 
        % Take advantage of the already applied Pygments mark-up to insert 
        % potential linebreaks for TeX processing. 
        %        {, <, #, %, $, ' and ": go to next line. 
        %        _, }, ^, &, >, - and ~: stay at end of broken line. 
        % Use of \textquotesingle for straight quote. 
        \newcommand*\Wrappedbreaksatspecials {% 
            \def\PYGZus{\discretionary{\char`\_}{\Wrappedafterbreak}{\char`\_}}% 
            \def\PYGZob{\discretionary{}{\Wrappedafterbreak\char`\{}{\char`\{}}% 
            \def\PYGZcb{\discretionary{\char`\}}{\Wrappedafterbreak}{\char`\}}}% 
            \def\PYGZca{\discretionary{\char`\^}{\Wrappedafterbreak}{\char`\^}}% 
            \def\PYGZam{\discretionary{\char`\&}{\Wrappedafterbreak}{\char`\&}}% 
            \def\PYGZlt{\discretionary{}{\Wrappedafterbreak\char`\<}{\char`\<}}% 
            \def\PYGZgt{\discretionary{\char`\>}{\Wrappedafterbreak}{\char`\>}}% 
            \def\PYGZsh{\discretionary{}{\Wrappedafterbreak\char`\#}{\char`\#}}% 
            \def\PYGZpc{\discretionary{}{\Wrappedafterbreak\char`\%}{\char`\%}}% 
            \def\PYGZdl{\discretionary{}{\Wrappedafterbreak\char`\$}{\char`\$}}% 
            \def\PYGZhy{\discretionary{\char`\-}{\Wrappedafterbreak}{\char`\-}}% 
            \def\PYGZsq{\discretionary{}{\Wrappedafterbreak\textquotesingle}{\textquotesingle}}% 
            \def\PYGZdq{\discretionary{}{\Wrappedafterbreak\char`\"}{\char`\"}}% 
            \def\PYGZti{\discretionary{\char`\~}{\Wrappedafterbreak}{\char`\~}}% 
        } 
        % Some characters . , ; ? ! / are not pygmentized. 
        % This macro makes them "active" and they will insert potential linebreaks 
        \newcommand*\Wrappedbreaksatpunct {% 
            \lccode`\~`\.\lowercase{\def~}{\discretionary{\hbox{\char`\.}}{\Wrappedafterbreak}{\hbox{\char`\.}}}% 
            \lccode`\~`\,\lowercase{\def~}{\discretionary{\hbox{\char`\,}}{\Wrappedafterbreak}{\hbox{\char`\,}}}% 
            \lccode`\~`\;\lowercase{\def~}{\discretionary{\hbox{\char`\;}}{\Wrappedafterbreak}{\hbox{\char`\;}}}% 
            \lccode`\~`\:\lowercase{\def~}{\discretionary{\hbox{\char`\:}}{\Wrappedafterbreak}{\hbox{\char`\:}}}% 
            \lccode`\~`\?\lowercase{\def~}{\discretionary{\hbox{\char`\?}}{\Wrappedafterbreak}{\hbox{\char`\?}}}% 
            \lccode`\~`\!\lowercase{\def~}{\discretionary{\hbox{\char`\!}}{\Wrappedafterbreak}{\hbox{\char`\!}}}% 
            \lccode`\~`\/\lowercase{\def~}{\discretionary{\hbox{\char`\/}}{\Wrappedafterbreak}{\hbox{\char`\/}}}% 
            \catcode`\.\active
            \catcode`\,\active 
            \catcode`\;\active
            \catcode`\:\active
            \catcode`\?\active
            \catcode`\!\active
            \catcode`\/\active 
            \lccode`\~`\~ 	
        }
    \makeatother

    \let\OriginalVerbatim=\Verbatim
    \makeatletter
    \renewcommand{\Verbatim}[1][1]{%
        %\parskip\z@skip
        \sbox\Wrappedcontinuationbox {\Wrappedcontinuationsymbol}%
        \sbox\Wrappedvisiblespacebox {\FV@SetupFont\Wrappedvisiblespace}%
        \def\FancyVerbFormatLine ##1{\hsize\linewidth
            \vtop{\raggedright\hyphenpenalty\z@\exhyphenpenalty\z@
                \doublehyphendemerits\z@\finalhyphendemerits\z@
                \strut ##1\strut}%
        }%
        % If the linebreak is at a space, the latter will be displayed as visible
        % space at end of first line, and a continuation symbol starts next line.
        % Stretch/shrink are however usually zero for typewriter font.
        \def\FV@Space {%
            \nobreak\hskip\z@ plus\fontdimen3\font minus\fontdimen4\font
            \discretionary{\copy\Wrappedvisiblespacebox}{\Wrappedafterbreak}
            {\kern\fontdimen2\font}%
        }%
        
        % Allow breaks at special characters using \PYG... macros.
        \Wrappedbreaksatspecials
        % Breaks at punctuation characters . , ; ? ! and / need catcode=\active 	
        \OriginalVerbatim[#1,codes*=\Wrappedbreaksatpunct]%
    }
    \makeatother

    % Exact colors from NB
    \definecolor{incolor}{HTML}{303F9F}
    \definecolor{outcolor}{HTML}{D84315}
    \definecolor{cellborder}{HTML}{CFCFCF}
    \definecolor{cellbackground}{HTML}{F7F7F7}
    
    % prompt
    \makeatletter
    \newcommand{\boxspacing}{\kern\kvtcb@left@rule\kern\kvtcb@boxsep}
    \makeatother
    \newcommand{\prompt}[4]{
        {\ttfamily\llap{{\color{#2}[#3]:\hspace{3pt}#4}}\vspace{-\baselineskip}}
    }
    
    \pagestyle{fancy}

    
    % Prevent overflowing lines due to hard-to-break entities
    \sloppy 
    % Setup hyperref package
    \hypersetup{
      breaklinks=true,  % so long urls are correctly broken across lines
      colorlinks=true,
      urlcolor=urlcolor,
      linkcolor=linkcolor,
      citecolor=citecolor,
      }
    % Slightly bigger margins than the latex defaults
    
    \geometry{verbose,tmargin=1in,bmargin=1in,lmargin=1in,rmargin=1in}
    
    

\begin{document}
    
    %-------------------------------
%	TITLE SECTION
%-------------------------------

\fancyhead[C]{}
\hrule \medskip % Upper rule
\begin{minipage}{0.295\textwidth}
\raggedright
\footnotesize
Francisco Javier Sáez Maldonado \hfill\\
77448344F \hfill\\
franciscojavier.saez@estudiante.uam.es
\end{minipage}
\begin{minipage}{0.4\textwidth}
\centering
\large
Ejercicios Estimación P.D.F.\\
\normalsize
Métodos Avanzados en Estadística\\
\end{minipage}
\begin{minipage}{0.295\textwidth}
\raggedleft
\today\hfill\\
\end{minipage}
\medskip\hrule
\bigskip

%-------------------------------
%	CONTENTS
%-------------------------------
    
    \begin{problem}{6}

    
    Sean \(X_1,\dots,X_n\) v.a.i.i.d. de una distribución con densidad
\(f\). Se considera el estimador del núcleo \(\hat{f}\) con núcleo
rectangular \(K(x) = \mathbb I_{[-1/2,1/2]}(x)\) y parámetro de
suavizado \(h\).

\begin{enumerate}
\item Calcula el sesgo y la varianza de \(\hat f(x)\) para un valor de \(x\)
fijo.

\item Demuestra que tanto el sesgo como la varianza tienden a cero si
\(h\to 0\) y \(nh\to \infty\)
\end{enumerate}

    \end{problem}

Tenemos que calcular \(E\left[\hat{f}(x)\right]\). Vamos primero a
expresar el núcleo de otra manera. Tenemos que tratar con $ K\left(
\frac{x - X_i}{h}\right)$, siendo \(K\) la función indicadora en el
intervalo \((-1/2,1/2)\). Podemos ver entonces que \[
 K\left( \frac{x - X_i}{h}\right) = \mathbb I_{(-1/2,1/2)} \left( \frac{x - X_i}{h}\right) = 
 \begin{cases}
 1 & \left| \frac{x - X_i}{h} \right| < \frac{1}{2}\\
 0 & \text{else}
 \end{cases}
\] Por lo que, si despejamos del miembro superior de la última igualdad,
obtenemos que esta función será \(1\) cuando: \[
-\frac{h}{2} < x - X_i < \frac{h}{2} \Longleftrightarrow x - \frac{h}{2} < \lvert X_i \rvert <  x +\frac{h}{2} 
\]

Vemos que \begin{align*}
E\left[\hat{f}(x)\right] & = E \left[ \frac{1}{nh} \sum K\left( \frac{x - X_i}{h}\right)\right]\\
 & = \frac{1}{nh}  \sum  E \left[  K\left( \frac{X_i - t}{h}\right) \right] \\
 &= \frac{1}{nh}  \sum \int  K\left( \frac{X_i - t}{h}\right) f(t) dt \\
 & = \frac{1}{nh} \sum \int_{-\frac{h}{2} - x}^{\frac{h}{2}  + x} f(t) dt\\
 & =  \frac{1}{nh} \sum  P\left(x -\frac{h}{2}  \leq X_i \leq x + \frac{h}{2}\right)
\end{align*} Ahora, como estas probabilidades son la misma para
cualquiera de las \(X_i\) y sabiendo que los miembros de la sumatoria
son la definición de la función de distribución, podemos escribir eso
como: \[
E\left[\hat{f}(x)\right] = \frac{F(x + \frac{h}{2}) - F(x -\frac{h}{2})}{h}
\]

Con esto, podemos decir que \[
Bias(\hat{f}(x)) = E\left[\hat{f}(x)\right] - f(x) = \frac{F(x + \frac{h}{2}) - F(x -\frac{h}{2})}{h} - f(x)
\]

Vamos a calcular ahora la varianza de \(\hat{f}(x)\). Sabemos que \[
Var(\hat{f}(x)) = Var\left(\frac{1}{nh} \sum K\left( \frac{x - X_i}{h}\right)\right) = \frac{1}{n^2 h^2} \sum Var\left(K\left( \frac{x - X_i}{h}\right)\right) = \frac{n}{n^2 h^2} Var\left(K\left( \frac{x - X_i}{h}\right)\right)
\] Donde, en la última igualdad hemos usado que la varianza para todas
las variables es la misma pues siguen la misma \(f\). Calculemos
entonces \(Var\left(K\left( \frac{x - X_i}{h}\right)\right)\). \[
Var\left(K\left( \frac{x - X_i}{h}\right)\right) = E\left[K\left( \frac{x - X_i}{h}\right)^2\right] - E\left[K\left( \frac{x - X_i}{h}\right)\right]^2
\] Pero, sabemos que al ser una función indicadora, \(K(x)^2 = K(x)\),
por lo que tenemos (usando lo que hemos calculado en el ejercicio
anterior): \begin{align*}
Var\left(K\left( \frac{x - X_i}{h}\right)\right) & = E\left[K\left( \frac{x - X_i}{h}\right)\right] - E\left[K\left( \frac{x - X_i}{h}\right)\right]^2 \\
& = P\left(x -\frac{h}{2}  \leq X_i \leq x + \frac{h}{2}\right) - P\left(x -\frac{h}{2}  \leq X_i \leq x + \frac{h}{2}\right)^2\\
& = \left(F\left(x + \frac{h}{2}\right) - F\left(x -\frac{h}{2}\right)\right)  \left(1 - \left(F(x + \frac{h}{2}) - F(x -\frac{h}{2})\right)\right)
\end{align*} Por lo que \[
Var(\hat{f}(x)) = \frac{\left(F\left(x + \frac{h}{2}\right) - F\left(x -\frac{h}{2}\right)\right)  \left(1 - \left(F(x + \frac{h}{2}) - F(x -\frac{h}{2})\right)\right)}{n h^2}
\]

Ahora, nos queda evaluar los límites cuando \(h\to 0\) y cuando
\(nh\to 0\). En el primer caso, basta ver que: \begin{align*}
\lim_{h\to 0}  \frac{F(x + \frac{h}{2}) - F(x -\frac{h}{2})}{h}& = \frac{1}{2}\left(\lim_{h\to 0} \frac{F(x+h/2) - F(x) + F(x) - F(x-h/2)}{h}\right) \\
 & = \frac{1}{2}\left(\lim_{h/2\to 0}\frac{F(x+h/2) - F(x)}{h/2} - \lim_{h/2\to 0}\frac{F(x-h/2) - F(x)}{-h/2}\right)\\
&= \frac{2 F'(x)}{2}\\
& = f(x)
\end{align*}

Por tanto, tenemos que \[
\lim_{h\to 0} Bias(\hat{f}(x)) =  \lim_{h\to 0}\left(\frac{F(x + \frac{h}{2}) - F(x -\frac{h}{2})}{h} - f(x)\right) = \lim_{h\to 0} \left(f(x) - f(x)\right) = 0
\] Ahora, hacemos el mismo proceso para la varianza:

\begin{align*}
\lim_{\substack{nh \to \infty \\ h \to 0}} Var(\hat{f}(x))  &= \lim_{\substack{nh \to \infty \\ h \to 0}}\frac{ \left(F\left(x + \frac{h}{2}\right) - F\left(x -\frac{h}{2}\right)\right)  \left(1 - \left(F(x + \frac{h}{2}) - F(x -\frac{h}{2})\right)\right)}{n h^2} \\
& = \lim_{\substack{nh \to \infty \\ h \to 0}} \left( \frac{1}{nh} f(x)\left(1 - \left(F(x + \frac{h}{2}) - F(x -\frac{h}{2})\right)\right)\right)\\
& =  \lim_{\substack{nh \to \infty \\ h \to 0}} \left( \frac{1}{nh} f(x)\right)\\
& = 0
\end{align*}

\begin{problem}{7}
    Considera una variable aleatoria con distribución beta de parámetros $\alpha = 3,\beta=6$.
    \begin{enumerate}
\item Representa gráficamente la función de densidad y la función de distribución.
\item Simula una muestra de tamaño $20$ de la distribución. A continuación, representa en los mismos gráficos del apartado $(1)$ las estimaciones de $F$ y $f$ obtenidas respectivamente mediante la función de distribución empírica $F_n$ y un estimador del núcleo $\hat f$ obtenidos a partir de la muestra simulada.
\item Verifica empíricamente el grado de aproximación alcanzado en las estimaciones de $F$ y $f$. Para ello, genera 200 muestras de tamaño $20$ y, para cada una de ellas, evalúa el error (medido en la norma del supremo, es deccir, el máximo de las diferencias entre las funciones) cometido al aproximar $F$ por $F_n$ y $f$ por $\hat f$. Por último, calcula le promedio de los $200$ errores cometidos.
    \end{enumerate}

\end{problem}

    Incluimos las librerías y opciones necesarias para el correcto
funcionamiento del ejercicio.

    \begin{tcolorbox}[breakable, size=fbox, boxrule=1pt, pad at break*=1mm,colback=cellbackground, colframe=cellborder]
\prompt{In}{incolor}{1}{\boxspacing}
\begin{Verbatim}[commandchars=\\\{\}]
\PY{n+nf}{suppressMessages}\PY{p}{(}\PY{n+nf}{library}\PY{p}{(}\PY{n}{tidyverse}\PY{p}{)}\PY{p}{)}
\PY{n+nf}{suppressMessages}\PY{p}{(}\PY{n+nf}{library}\PY{p}{(}\PY{n}{reshape2}\PY{p}{)}\PY{p}{)}
\PY{n}{defaultW} \PY{o}{\PYZlt{}\PYZhy{}} \PY{n+nf}{getOption}\PY{p}{(}\PY{l+s}{\PYZdq{}}\PY{l+s}{warn\PYZdq{}}\PY{p}{)}
\PY{n+nf}{options}\PY{p}{(}\PY{n}{warn} \PY{o}{=} \PY{l+m}{\PYZhy{}1}\PY{p}{)}
\PY{n+nf}{library}\PY{p}{(}\PY{n}{patchwork}\PY{p}{)}
\PY{n+nf}{options}\PY{p}{(}\PY{n}{repr.plot.width}\PY{o}{=}\PY{l+m}{12}\PY{p}{,} \PY{n}{repr.plot.height}\PY{o}{=}\PY{l+m}{8}\PY{p}{)}
\PY{n+nf}{set.seed}\PY{p}{(}\PY{l+m}{123}\PY{p}{)}
\end{Verbatim}
\end{tcolorbox}

    Representamos de forma gráfica la p.d.f. y la función de distribución.

Fijamos los valores \(\alpha\) y \(\beta\) para la distribución beta.

    \begin{tcolorbox}[breakable, size=fbox, boxrule=1pt, pad at break*=1mm,colback=cellbackground, colframe=cellborder]
\prompt{In}{incolor}{2}{\boxspacing}
\begin{Verbatim}[commandchars=\\\{\}]
\PY{n}{alpha} \PY{o}{\PYZlt{}\PYZhy{}} \PY{l+m}{3}
\PY{n}{beta} \PY{o}{\PYZlt{}\PYZhy{}} \PY{l+m}{6}
\end{Verbatim}
\end{tcolorbox}

    \begin{tcolorbox}[breakable, size=fbox, boxrule=1pt, pad at break*=1mm,colback=cellbackground, colframe=cellborder]
\prompt{In}{incolor}{3}{\boxspacing}
\begin{Verbatim}[commandchars=\\\{\}]
\PY{n}{plot1} \PY{o}{\PYZlt{}\PYZhy{}} \PY{n+nf}{ggplot}\PY{p}{(}\PY{k+kc}{NULL}\PY{p}{)} \PY{o}{+} \PY{n+nf}{stat\PYZus{}function}\PY{p}{(}\PY{n}{fun} \PY{o}{=} \PY{n+nf}{function}\PY{p}{(}\PY{n}{x}\PY{p}{)} \PY{n+nf}{pbeta}\PY{p}{(}\PY{n}{x}\PY{p}{,} \PY{n}{alpha}\PY{p}{,} \PY{n}{beta}\PY{p}{)}\PY{p}{,} \PY{n+nf}{aes}\PY{p}{(}\PY{n}{colour} \PY{o}{=} \PY{l+s}{\PYZdq{}}\PY{l+s}{Beta CDF\PYZdq{}}\PY{p}{)}\PY{p}{,}\PY{n}{size} \PY{o}{=} \PY{l+m}{1}\PY{p}{)} \PY{o}{+} \PY{n+nf}{scale\PYZus{}color\PYZus{}manual}\PY{p}{(}\PY{l+s}{\PYZdq{}}\PY{l+s}{Legend\PYZdq{}}\PY{p}{,}\PY{n}{values}\PY{o}{=}\PY{n+nf}{c}\PY{p}{(}\PY{l+s}{\PYZdq{}}\PY{l+s}{blue\PYZdq{}}\PY{p}{)}\PY{p}{)}
\PY{n}{plot2} \PY{o}{\PYZlt{}\PYZhy{}} \PY{n+nf}{ggplot}\PY{p}{(}\PY{k+kc}{NULL}\PY{p}{)} \PY{o}{+} \PY{n+nf}{stat\PYZus{}function}\PY{p}{(}\PY{n}{fun} \PY{o}{=} \PY{n+nf}{function}\PY{p}{(}\PY{n}{x}\PY{p}{)} \PY{n+nf}{dbeta}\PY{p}{(}\PY{n}{x}\PY{p}{,} \PY{n}{alpha}\PY{p}{,} \PY{n}{beta}\PY{p}{)}\PY{p}{,} \PY{n+nf}{aes}\PY{p}{(}\PY{n}{colour} \PY{o}{=} \PY{l+s}{\PYZdq{}}\PY{l+s}{Beta PDF\PYZdq{}}\PY{p}{)}\PY{p}{,}\PY{n}{size} \PY{o}{=} \PY{l+m}{1}\PY{p}{)} \PY{o}{+} \PY{n+nf}{scale\PYZus{}color\PYZus{}manual}\PY{p}{(}\PY{l+s}{\PYZdq{}}\PY{l+s}{Legend\PYZdq{}}\PY{p}{,}\PY{n}{values}\PY{o}{=}\PY{n+nf}{c}\PY{p}{(}\PY{l+s}{\PYZdq{}}\PY{l+s}{red\PYZdq{}}\PY{p}{)}\PY{p}{)}
\PY{n}{plot1} \PY{o}{+} \PY{n}{plot2} \PY{o}{+} \PY{n+nf}{plot\PYZus{}layout}\PY{p}{(}\PY{n}{widths} \PY{o}{=} \PY{n+nf}{c}\PY{p}{(}\PY{l+m}{3}\PY{p}{,} \PY{l+m}{3}\PY{p}{)}\PY{p}{)}
\end{Verbatim}
\end{tcolorbox}

    \begin{center}
    \adjustimage{max size={0.9\linewidth}{0.9\paperheight}}{output_5_0.png}
    \end{center}
    { \hspace*{\fill} \\}
    
    Vamos con el \textbf{segundo apartado}. Vamos a extraer de la
distribución beta una muestra de tamaño \(20\), hallaremos el estimador
del núcleo y dibujaremos tanto la función de densidad como el estimador
del núcleo que nos da la función \texttt{density} de \texttt{R}.

    \begin{tcolorbox}[breakable, size=fbox, boxrule=1pt, pad at break*=1mm,colback=cellbackground, colframe=cellborder]
\prompt{In}{incolor}{4}{\boxspacing}
\begin{Verbatim}[commandchars=\\\{\}]
\PY{n}{n} \PY{o}{\PYZlt{}\PYZhy{}} \PY{l+m}{20}
\PY{c+c1}{\PYZsh{} Extract sample}
\PY{n}{sample} \PY{o}{\PYZlt{}\PYZhy{}} \PY{n+nf}{rbeta}\PY{p}{(}\PY{n}{n}\PY{p}{,}\PY{n}{alpha}\PY{p}{,}\PY{n}{beta}\PY{p}{)}
\PY{c+c1}{\PYZsh{} Kernel estimator}
\PY{n}{kernel\PYZus{}estimator} \PY{o}{\PYZlt{}\PYZhy{}} \PY{n+nf}{density}\PY{p}{(}\PY{n}{sample}\PY{p}{)}

\PY{n+nf}{ggplot}\PY{p}{(}\PY{k+kc}{NULL}\PY{p}{)} \PY{o}{+}
    \PY{n+nf}{stat\PYZus{}function}\PY{p}{(}\PY{n}{fun}\PY{o}{=}\PY{n}{dbeta}\PY{p}{,}\PY{n+nf}{aes}\PY{p}{(}\PY{n}{colour}\PY{o}{=}\PY{l+s}{\PYZsq{}}\PY{l+s}{P.D.F.\PYZsq{}}\PY{p}{)}\PY{p}{,}\PY{n}{args}\PY{o}{=}\PY{n+nf}{list}\PY{p}{(}\PY{n}{shape1}\PY{o}{=}\PY{n}{alpha}\PY{p}{,}\PY{n}{shape2}\PY{o}{=}\PY{n}{beta}\PY{p}{)}\PY{p}{,}\PY{n}{size}\PY{o}{=}\PY{l+m}{1}\PY{p}{)}\PY{o}{+}
    \PY{n+nf}{geom\PYZus{}line}\PY{p}{(}\PY{n+nf}{aes}\PY{p}{(}\PY{n}{x}\PY{o}{=}\PY{n}{kernel\PYZus{}estimator}\PY{o}{\PYZdl{}}\PY{n}{x}\PY{p}{,}\PY{n}{y}\PY{o}{=}\PY{n}{kernel\PYZus{}estimator}\PY{o}{\PYZdl{}}\PY{n}{y}\PY{p}{,}\PY{n}{colour}\PY{o}{=}\PY{l+s}{\PYZdq{}}\PY{l+s}{Kernel\PYZhy{}estimated pdf\PYZdq{}}\PY{p}{)}\PY{p}{,}\PY{n}{size}\PY{o}{=}\PY{l+m}{1}\PY{p}{)}\PY{o}{+}
    \PY{n+nf}{scale\PYZus{}color\PYZus{}manual}\PY{p}{(}\PY{l+s}{\PYZdq{}}\PY{l+s}{Legend\PYZdq{}}\PY{p}{,}\PY{n}{values}\PY{o}{=}\PY{n+nf}{c}\PY{p}{(}\PY{l+s}{\PYZdq{}}\PY{l+s}{blue\PYZdq{}}\PY{p}{,}\PY{l+s}{\PYZdq{}}\PY{l+s}{red\PYZdq{}}\PY{p}{)}\PY{p}{)}\PY{o}{+}
    \PY{n+nf}{xlim}\PY{p}{(}\PY{l+m}{\PYZhy{}0.01}\PY{p}{,}\PY{l+m}{1.01}\PY{p}{)}
\end{Verbatim}
\end{tcolorbox}

    \begin{center}
    \adjustimage{max size={0.9\linewidth}{0.9\paperheight}}{output_7_0.png}
    \end{center}
    { \hspace*{\fill} \\}
    
    A continuación, dibujamos la función de distribución teórica
comparándola con la \(F_n\) empírica obtenida por la muestra anterior.

    \begin{tcolorbox}[breakable, size=fbox, boxrule=1pt, pad at break*=1mm,colback=cellbackground, colframe=cellborder]
\prompt{In}{incolor}{5}{\boxspacing}
\begin{Verbatim}[commandchars=\\\{\}]
\PY{n}{x} \PY{o}{\PYZlt{}\PYZhy{}} \PY{n+nf}{seq}\PY{p}{(}\PY{l+m}{0}\PY{p}{,} \PY{l+m}{1}\PY{p}{,} \PY{l+m}{0.01}\PY{p}{)} 
\PY{n}{orig} \PY{o}{\PYZlt{}\PYZhy{}} \PY{n+nf}{pbeta}\PY{p}{(}\PY{n}{x}\PY{p}{,} \PY{n}{shape1} \PY{o}{=} \PY{n}{alpha}\PY{p}{,} \PY{n}{shape2} \PY{o}{=} \PY{n}{beta}\PY{p}{)}
\PY{n}{data} \PY{o}{\PYZlt{}\PYZhy{}} \PY{n+nf}{data.frame}\PY{p}{(} \PY{n}{x} \PY{o}{=} \PY{n}{sample}\PY{p}{)}
\PY{n+nf}{ggplot}\PY{p}{(}\PY{k+kc}{NULL}\PY{p}{)}\PY{o}{+}
    \PY{n+nf}{stat\PYZus{}ecdf}\PY{p}{(}\PY{n}{data} \PY{o}{=} \PY{n}{data}\PY{p}{,}\PY{n+nf}{aes}\PY{p}{(}\PY{n}{x}\PY{p}{,}\PY{n}{colour}\PY{o}{=}\PY{l+s}{\PYZdq{}}\PY{l+s}{Estimated C.D.F\PYZdq{}}\PY{p}{)}\PY{p}{)} \PY{o}{+}
    \PY{n+nf}{stat\PYZus{}ecdf}\PY{p}{(}\PY{n}{data} \PY{o}{=} \PY{n}{data}\PY{p}{,}\PY{n+nf}{aes}\PY{p}{(}\PY{n}{x}\PY{p}{)}\PY{p}{,} \PY{n}{geom} \PY{o}{=} \PY{l+s}{\PYZsq{}}\PY{l+s}{point\PYZsq{}}\PY{p}{)} \PY{o}{+}
    \PY{n+nf}{geom\PYZus{}line}\PY{p}{(}\PY{n+nf}{aes}\PY{p}{(}\PY{n}{x}\PY{p}{,}\PY{n}{orig}\PY{p}{,}\PY{n}{colour}\PY{o}{=}\PY{l+s}{\PYZdq{}}\PY{l+s}{Original CDF\PYZdq{}}\PY{p}{)}\PY{p}{,}\PY{n}{size}\PY{o}{=}\PY{l+m}{1}\PY{p}{)}\PY{o}{+}
    \PY{n+nf}{scale\PYZus{}color\PYZus{}manual}\PY{p}{(}\PY{l+s}{\PYZdq{}}\PY{l+s}{Legend\PYZdq{}}\PY{p}{,}\PY{n}{values}\PY{o}{=}\PY{n+nf}{c}\PY{p}{(}\PY{l+s}{\PYZdq{}}\PY{l+s}{blue\PYZdq{}}\PY{p}{,}\PY{l+s}{\PYZdq{}}\PY{l+s}{red\PYZdq{}}\PY{p}{)}\PY{p}{)}
\end{Verbatim}
\end{tcolorbox}

    \begin{center}
    \adjustimage{max size={0.9\linewidth}{0.9\paperheight}}{output_9_0.png}
    \end{center}
    { \hspace*{\fill} \\}
    
    Por último, realizamos el \textbf{tercer apartado}. Vamos a realizar
\(200\) experimentos en los que extraemos \(n=20\) muestras y, para cada
una de ellas calcularemos el error entre el estimador del núcleo de la
muestra y la función de densidad teórica. Haremos lo mismo con la
función de distribución.

    \begin{tcolorbox}[breakable, size=fbox, boxrule=1pt, pad at break*=1mm,colback=cellbackground, colframe=cellborder]
\prompt{In}{incolor}{6}{\boxspacing}
\begin{Verbatim}[commandchars=\\\{\}]
\PY{c+c1}{\PYZsh{} Initialize variables}
\PY{n}{num\PYZus{}experiments} \PY{o}{\PYZlt{}\PYZhy{}} \PY{l+m}{200}
\PY{n}{experiment\PYZus{}size} \PY{o}{\PYZlt{}\PYZhy{}} \PY{l+m}{20}
\PY{n}{errors\PYZus{}f} \PY{o}{\PYZlt{}\PYZhy{}} \PY{n+nf}{rep}\PY{p}{(}\PY{k+kc}{NA}\PY{p}{,}\PY{n}{num\PYZus{}experiments}\PY{p}{)}
\PY{n}{p\PYZus{}values\PYZus{}f} \PY{o}{\PYZlt{}\PYZhy{}} \PY{n+nf}{rep}\PY{p}{(}\PY{k+kc}{NA}\PY{p}{,}\PY{n}{num\PYZus{}experiments}\PY{p}{)}
\PY{n}{errors\PYZus{}F} \PY{o}{\PYZlt{}\PYZhy{}} \PY{n+nf}{rep}\PY{p}{(}\PY{k+kc}{NA}\PY{p}{,}\PY{n}{num\PYZus{}experiments}\PY{p}{)}
\PY{n}{p\PYZus{}values\PYZus{}F} \PY{o}{\PYZlt{}\PYZhy{}} \PY{n+nf}{rep}\PY{p}{(}\PY{k+kc}{NA}\PY{p}{,}\PY{n}{num\PYZus{}experiments}\PY{p}{)}
\PY{c+c1}{\PYZsh{} Repeat experiment}
\PY{n+nf}{for }\PY{p}{(}\PY{n}{i} \PY{n}{in} \PY{l+m}{0}\PY{o}{:}\PY{n}{num\PYZus{}experiments}\PY{p}{)}\PY{p}{\PYZob{}}
    
    \PY{c+c1}{\PYZsh{} Obtain sample, kernel estimator, and evaluate the xs obtained in the estimator}
    \PY{n}{sample} \PY{o}{\PYZlt{}\PYZhy{}} \PY{n+nf}{rbeta}\PY{p}{(}\PY{n}{experiment\PYZus{}size}\PY{p}{,}\PY{n}{alpha}\PY{p}{,}\PY{n}{beta}\PY{p}{)}
    \PY{c+c1}{\PYZsh{} Kernel estimator from the sample}
    \PY{n}{kernel\PYZus{}estimator} \PY{o}{\PYZlt{}\PYZhy{}} \PY{n+nf}{density}\PY{p}{(}\PY{n}{sample}\PY{p}{)}
    \PY{c+c1}{\PYZsh{} Obtain theoretical density values in the x coordinates}
    \PY{n}{true\PYZus{}fs} \PY{o}{\PYZlt{}\PYZhy{}} \PY{n+nf}{dbeta}\PY{p}{(}\PY{n}{kernel\PYZus{}estimator}\PY{o}{\PYZdl{}}\PY{n}{x}\PY{p}{,}\PY{n}{alpha}\PY{p}{,}\PY{n}{beta}\PY{p}{)}
    
    \PY{c+c1}{\PYZsh{} Obtain the empirical CDF and evaluate in the sample}
    \PY{n}{empirical\PYZus{}F} \PY{o}{\PYZlt{}\PYZhy{}} \PY{n+nf}{ecdf}\PY{p}{(}\PY{n}{sample}\PY{p}{)}
    \PY{c+c1}{\PYZsh{}Obtain theoretical distribution function values in x coordinates}
    \PY{n}{true\PYZus{}Fs} \PY{o}{\PYZlt{}\PYZhy{}} \PY{n+nf}{pbeta}\PY{p}{(}\PY{n}{kernel\PYZus{}estimator}\PY{o}{\PYZdl{}}\PY{n}{x}\PY{p}{,}\PY{n}{alpha}\PY{p}{,}\PY{n}{beta}\PY{p}{)}
    \PY{c+c1}{\PYZsh{} Obtain empirical F in x coordinates}
    \PY{n}{Fs} \PY{o}{\PYZlt{}\PYZhy{}} \PY{n+nf}{empirical\PYZus{}F}\PY{p}{(}\PY{n}{kernel\PYZus{}estimator}\PY{o}{\PYZdl{}}\PY{n}{x}\PY{p}{)}
    
    \PY{c+c1}{\PYZsh{} Compute the KS test (provides distance and pvalue)}
    \PY{n}{test\PYZus{}f} \PY{o}{\PYZlt{}\PYZhy{}} \PY{n+nf}{ks.test}\PY{p}{(}\PY{n}{kernel\PYZus{}estimator}\PY{o}{\PYZdl{}}\PY{n}{y}\PY{p}{,} \PY{n}{true\PYZus{}fs}\PY{p}{)}
    \PY{n}{test\PYZus{}F} \PY{o}{\PYZlt{}\PYZhy{}} \PY{n+nf}{ks.test}\PY{p}{(}\PY{n}{Fs}\PY{p}{,}\PY{n}{true\PYZus{}Fs}\PY{p}{)}
    
    \PY{c+c1}{\PYZsh{} Save the result}
    \PY{n}{errors\PYZus{}f}\PY{p}{[}\PY{n}{i}\PY{l+m}{+1}\PY{p}{]} \PY{o}{\PYZlt{}\PYZhy{}} \PY{n}{test\PYZus{}f}\PY{o}{\PYZdl{}}\PY{n}{statistic}
    \PY{n}{errors\PYZus{}F}\PY{p}{[}\PY{n}{i}\PY{l+m}{+1}\PY{p}{]} \PY{o}{\PYZlt{}\PYZhy{}} \PY{n}{test\PYZus{}F}\PY{o}{\PYZdl{}}\PY{n}{statistic}
    
    \PY{n}{p\PYZus{}values\PYZus{}f}\PY{p}{[}\PY{n}{i}\PY{l+m}{+1}\PY{p}{]} \PY{o}{\PYZlt{}\PYZhy{}} \PY{n}{test\PYZus{}f}\PY{o}{\PYZdl{}}\PY{n}{p.value}
    \PY{n}{p\PYZus{}values\PYZus{}F}\PY{p}{[}\PY{n}{i}\PY{l+m}{+1}\PY{p}{]} \PY{o}{\PYZlt{}\PYZhy{}} \PY{n}{test\PYZus{}F}\PY{o}{\PYZdl{}}\PY{n}{p.value}

\PY{p}{\PYZcb{}}
\end{Verbatim}
\end{tcolorbox}

    Podemos ahora mostrar la información que hemos obtenido. Mediante la
función \texttt{summary}, mostramos un resumen de la misma.

    \begin{tcolorbox}[breakable, size=fbox, boxrule=1pt, pad at break*=1mm,colback=cellbackground, colframe=cellborder]
\prompt{In}{incolor}{7}{\boxspacing}
\begin{Verbatim}[commandchars=\\\{\}]
\PY{n+nf}{print}\PY{p}{(}\PY{l+s}{\PYZdq{}}\PY{l+s}{Information about errors in the density function\PYZdq{}}\PY{p}{)}
\PY{n+nf}{summary}\PY{p}{(}\PY{n}{errors\PYZus{}f}\PY{p}{)}
\PY{n+nf}{print}\PY{p}{(}\PY{l+s}{\PYZdq{}}\PY{l+s}{p\PYZhy{}values in the density function\PYZdq{}}\PY{p}{)}
\PY{n+nf}{summary}\PY{p}{(}\PY{n}{p\PYZus{}values\PYZus{}f}\PY{p}{)}
\end{Verbatim}
\end{tcolorbox}

    \begin{Verbatim}[commandchars=\\\{\}]
[1] "Information about errors in the density function"
    \end{Verbatim}

    
    \begin{Verbatim}[commandchars=\\\{\}]
   Min. 1st Qu.  Median    Mean 3rd Qu.    Max. 
0.07617 0.14648 0.17969 0.18719 0.23047 0.34570 
    \end{Verbatim}

    
    \begin{Verbatim}[commandchars=\\\{\}]
[1] "p-values in the density function"
    \end{Verbatim}

    
    \begin{Verbatim}[commandchars=\\\{\}]
     Min.   1st Qu.    Median      Mean   3rd Qu.      Max. 
0.000e+00 0.000e+00 1.300e-07 1.777e-03 3.386e-05 1.025e-01 
    \end{Verbatim}

    
    \begin{tcolorbox}[breakable, size=fbox, boxrule=1pt, pad at break*=1mm,colback=cellbackground, colframe=cellborder]
\prompt{In}{incolor}{8}{\boxspacing}
\begin{Verbatim}[commandchars=\\\{\}]
\PY{n+nf}{print}\PY{p}{(}\PY{l+s}{\PYZdq{}}\PY{l+s}{Information about errors in the distribution function\PYZdq{}}\PY{p}{)}
\PY{n+nf}{summary}\PY{p}{(}\PY{n}{errors\PYZus{}F}\PY{p}{)}
\PY{n+nf}{print}\PY{p}{(}\PY{l+s}{\PYZdq{}}\PY{l+s}{p\PYZhy{}values in the distribution function\PYZdq{}}\PY{p}{)}
\PY{n+nf}{summary}\PY{p}{(}\PY{n}{p\PYZus{}values\PYZus{}F}\PY{p}{)}
\end{Verbatim}
\end{tcolorbox}

    \begin{Verbatim}[commandchars=\\\{\}]
[1] "Information about errors in the distribution function"
    \end{Verbatim}

    
    \begin{Verbatim}[commandchars=\\\{\}]
   Min. 1st Qu.  Median    Mean 3rd Qu.    Max. 
 0.1504  0.2031  0.2188  0.2170  0.2305  0.3184 
    \end{Verbatim}

    
    \begin{Verbatim}[commandchars=\\\{\}]
[1] "p-values in the distribution function"
    \end{Verbatim}

    
    \begin{Verbatim}[commandchars=\\\{\}]
     Min.   1st Qu.    Median      Mean   3rd Qu.      Max. 
0.000e+00 3.000e-12 4.600e-11 2.940e-07 1.338e-09 1.870e-05 
    \end{Verbatim}

    
    Podemos ver que la media de los errores es de \(0.18\) en la función de
densidad y de alrededor de \(0.21\) en la función de distribución. Vemos
que en ambos casos los p-values son prácticamente cero. Esto nos indica
que nuestro nivel de significancia es bastante alto, es decir, que
tenemos resultados significativos.



    % Add a bibliography block to the postdoc
    
    
    
\end{document}
