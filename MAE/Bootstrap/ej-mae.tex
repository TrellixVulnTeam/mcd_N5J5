\documentclass[a4paper]{article}

\addtolength{\hoffset}{-2.25cm}
\addtolength{\textwidth}{4.5cm}
\addtolength{\voffset}{-3.25cm}
\addtolength{\textheight}{5cm}
\setlength{\parskip}{0pt}
\setlength{\parindent}{0in}

%----------------------------------------------------------------------------------------
%	PACKAGES AND OTHER DOCUMENT CONFIGURATIONS
%----------------------------------------------------------------------------------------

\usepackage{blindtext} % Package to generate dummy text
\usepackage{charter} % Use the Charter font
\usepackage[utf8]{inputenc} % Use UTF-8 encoding
\usepackage{microtype} % Slightly tweak font spacing for aesthetics
\usepackage[english, ngerman]{babel} % Language hyphenation and typographical rules
\usepackage{amsthm, amsmath, amssymb} % Mathematical typesetting
\usepackage{float} % Improved interface for floating objects
\usepackage[final, colorlinks = true,
            linkcolor = black,
            citecolor = black]{hyperref} % For hyperlinks in the PDF
\usepackage{graphicx, multicol} % Enhanced support for graphics
\usepackage{xcolor} % Driver-independent color extensions
\usepackage{marvosym, wasysym} % More symbols
\usepackage{rotating} % Rotation tools
\usepackage{censor} % Facilities for controlling restricted text
\usepackage{listings} % Environment for non-formatted code, !uses style file!
\usepackage{pseudocode} % Environment for specifying algorithms in a natural way
 % Environment for f-structures, !uses style file!
\usepackage{booktabs} % Enhances quality of tables
\usepackage{tikz-qtree} % Easy tree drawing tool
 % Configuration for b-trees and b+-trees, !uses style file!
\usepackage[backend=biber,style=numeric,
            sorting=nyt]{biblatex} % Complete reimplementation of bibliographic facilities
\addbibresource{ecl.bib}
\usepackage{csquotes} % Context sensitive quotation facilities
\usepackage[yyyymmdd]{datetime} % Uses YEAR-MONTH-DAY format for dates
\renewcommand{\dateseparator}{-} % Sets dateseparator to '-'
\usepackage{fancyhdr} % Headers and footers
\pagestyle{fancy} % All pages have headers and footers
\fancyhead{}\renewcommand{\headrulewidth}{0pt} % Blank out the default header
\fancyfoot[L]{} % Custom footer text
\fancyfoot[C]{} % Custom footer text
\fancyfoot[R]{\thepage} % Custom footer text
\newcommand{\note}[1]{\marginpar{\scriptsize \textcolor{red}{#1}}} % Enables comments in red on margin
\usepackage{mathtools}
\usepackage{amsmath}
\DeclarePairedDelimiter\abs{\lvert}{\rvert}%
\usepackage{cancel}
%-------------------------------

%----------------------------------------------------------------------------------------

%-------------------------------
%	ENVIRONMENT SECTION
%-------------------------------
\pagestyle{fancy}
\usepackage{mdframed}


\newenvironment{problem}[2][Problema]
    { \begin{mdframed}[backgroundcolor=gray!20] \textbf{#1 #2} \\}
    {  \end{mdframed}}

% Define solution environment
\newenvironment{solution}
    {\textit{Solución}}
    {}


%-------------------------------------------------------------------------------------------
%	CUSTOM COMMANDS
%-------------------------------
\newcommand{\gaussian}{\frac{1}{\sigma\sqrt{2\pi}}\exp\left(- \frac{(x-\mu)^2}{2\sigma^2}\right)}
\newcommand{\R}{\mathbb R}



\begin{document}


%-------------------------------
%	TITLE SECTION
%-------------------------------

\fancyhead[C]{}
\hrule \medskip % Upper rule
\begin{minipage}{0.295\textwidth}
\raggedright
\footnotesize
Javier Sáez \hfill\\
77448344F \hfill\\
franciscojavier.saez@estudiante.uam.es
\end{minipage}
\begin{minipage}{0.4\textwidth}
\centering
\large
Ejercicios Bootstrap\\
\normalsize
Métodos Avanzados en Estadística\\
\end{minipage}
\begin{minipage}{0.295\textwidth}
\raggedleft
\today\hfill\\
\end{minipage}
\medskip\hrule
\bigskip

%-------------------------------
%	CONTENTS
%-------------------------------





\begin{problem}{2}
  Sea \( X_1,\dots,X_n\) una muestra de \(n\) observaciones iid de una distribución \(F\) con \(\mu\) y varianza \(\sigma^2\), y sea \(X_1^*,\dots,X_n^*\) una muestra de \(n\) observaciones iid de la distribución empírica de la muestra irignial \(F_n\). Calcula las siguientes cantidades:

  \begin{enumerate}
  \item \(E_{F_n}( \bar X_n^* ):= E(\bar X_n^* \ | X_1,\dots,X_n)\)
  \item \(E_F(\bar{X_n^*})\)
  \item \(Var_{F_n}\left(\bar X_n^*\right) := Var(\bar X_n^* \ | X_1,\dots, X_n)\)
  \item \(Var_F(\bar X_n^*)\)


    \end{enumerate}
  \end{problem}

   \begin{enumerate}

     \item \(E_{F_n}( \bar X_n^* ):= E(\bar X_n^* \ | X_1,\dots,X_n)\).\\

  Basta ver que, usando la definición y la linealidad de la esperanza,
  \[
  E_{F_n}\left( \bar{X^*_n}\right) = E_{F_n}\left[\frac{1}{n} \sum_{i=1}^n X_i^* \right] = \frac{1}{n} \sum_{i=1}^n E_{F_n} \left[X_i^*\right].
  \]
  Ahora, La esperanza bajo la función de distribución empírica de los \( X_i^*\) es la misma para todos los \(i\), porlo que podemos decir que estamos decir que estamos sumando \(n\) veces la esperanza de \(X_i^*\) habiendo fijado un \(i\). Tenemos por tanto:
  \[
  \frac{1}{n} \sum_{i=1}^n E_{F_n} \left[X_i^*\right] = E_{F_n}\left[X_i^*\right] = \sum_{x \in (X_1,\dots,X_n)} P(x) x  = \sum_x \frac{1}{n} x = \bar x
  \]

\item \(E_F(\bar{X_n^*})\).\\
  %% CUIDADO ESTE EJERCICIO ESTA ENTERO MAL

  Ahora no tenemos un condicionamiento como lo teníamos anteriormente, pero podemos usar la fórmula de la probabilidad total y ver que:
  \begin{align*}
    E_F(\bar{X_n^*}) = & E_F \left[ E_{F_n} \left( X_n^* | X_1,\dots, X_n\right) \right]\\
  \end{align*}

\item \(Var_{F_n}\left(\bar X_n^*\right) := Var(\bar X_n^* \ | X_1,\dots, X_n)\)

  \begin{align*}
    \stackrel{(1)}{=} & E_F[\bar Var_{F_n}\left(\frac{1}{n} \sum_{i=1}^n X_i^* \ | X_1,\dots,X_n\right)\\
     = & \frac{1}{n^2} \sum_{i=1}^n Var_{F_n}(X_i^* | X_1,\dots, X_n ) \\
     = & \frac{n}{n^2} Var_{F_n}(X_i^* | X_1,\dots,X_n),
  \end{align*}
  donde, en la última igualdad usamos que para cada una de las \(X_i^*\) la varianza bajo \(F_n\) es la misma, así que la estamos sumando \(n\) varianzas iguales. Calculamos ahora la varianza que 

\end{enumerate}
\end{document}
