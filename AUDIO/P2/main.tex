\documentclass[a4paper]{article}

\addtolength{\hoffset}{-2.25cm}
\addtolength{\textwidth}{4.5cm}
\addtolength{\voffset}{-3.25cm}
\addtolength{\textheight}{5cm}
\setlength{\parskip}{0pt}
\setlength{\parindent}{0in}

%----------------------------------------------------------------------------------------
%	PACKAGES AND OTHER DOCUMENT CONFIGURATIONS
%----------------------------------------------------------------------------------------

\usepackage{blindtext} % Package to generate dummy text
\usepackage{charter} % Use the Charter font
\usepackage[utf8]{inputenc} % Use UTF-8 encoding
\usepackage{microtype} % Slightly tweak font spacing for aesthetics
\usepackage[english]{babel} % Language hyphenation and typographical rules
\usepackage{amsthm, amsmath, amssymb} % Mathematical typesetting
\usepackage{float} % Improved interface for floating objects
\usepackage[final, colorlinks = true,
            linkcolor = black,
            citecolor = black]{hyperref} % For hyperlinks in the PDF
\usepackage{graphicx, multicol} % Enhanced support for graphics
\usepackage{xcolor} % Driver-independent color extensions
\usepackage{marvosym, wasysym} % More symbols
\usepackage{rotating} % Rotation tools
\usepackage{censor} % Facilities for controlling restricted text
\usepackage{listings} % Environment for non-formatted code, !uses style file!
\usepackage{pseudocode} % Environment for specifying algorithms in a natural way
 % Environment for f-structures, !uses style file!
\usepackage{booktabs} % Enhances quality of tables
\usepackage{tikz-qtree} % Easy tree drawing tool
 % Configuration for b-trees and b+-trees, !uses style file!
\usepackage[backend=biber,style=numeric,
            sorting=nyt]{biblatex} % Complete reimplementation of bibliographic facilities
\addbibresource{ecl.bib}
\usepackage{csquotes} % Context sensitive quotation facilities
\usepackage[yyyymmdd]{datetime} % Uses YEAR-MONTH-DAY format for dates
\renewcommand{\dateseparator}{-} % Sets dateseparator to '-'
\usepackage{fancyhdr} % Headers and footers
\pagestyle{fancy} % All pages have headers and footers
\fancyhead{}\renewcommand{\headrulewidth}{0pt} % Blank out the default header
\fancyfoot[L]{} % Custom footer text
\fancyfoot[C]{} % Custom footer text
\fancyfoot[R]{\thepage} % Custom footer text
\newcommand{\note}[1]{\marginpar{\scriptsize \textcolor{red}{#1}}} % Enables comments in red on margin
\usepackage{mathtools}
\usepackage{amsmath}
\DeclarePairedDelimiter\abs{\lvert}{\rvert}%
\usepackage{cancel}
\usepackage{minted}
\usepackage{float}
\usepackage{caption}
\usepackage{subcaption}
%-------------------------------

%----------------------------------------------------------------------------------------

%-------------------------------
%	ENVIRONMENT SECTION
%-------------------------------
\pagestyle{fancy}
\usepackage{mdframed}

\usepackage[sfdefault]{FiraSans} %% option 'sfdefault' activates Fira Sans as the default text font
\usepackage[T1]{fontenc}
\renewcommand*\oldstylenums[1]{{\firaoldstyle #1}}


% remove numbering from sections
\usepackage{titlesec}
\titleformat{\section}{\normalfont\Large\bfseries}{}{0pt}{}



%-------------------------------------------------------------------------------------------
%	CUSTOM COMMANDS
%-------------------------------
\newcommand{\gaussian}{\frac{1}{\sigma\sqrt{2\pi}}\exp\left(- \frac{(x-\mu)^2}{2\sigma^2}\right)}
\newcommand{\R}{\mathbb R}

\def\inline{\lstinline[basicstyle=\ttfamily,keywordstyle={}]}


\begin{document}


%-------------------------------
%	TITLE SECTION
%-------------------------------

\fancyhead[C]{}
\hrule \medskip % Upper rule
\begin{minipage}{0.295\textwidth}
  \raggedright
  \footnotesize
  Francisco Javier Sáez Maldonado \hfill\\
  franciscojavier.saez@estudiante.uam.es
  \hfill\\
\end{minipage}
\begin{minipage}{0.4\textwidth}
  \centering
  \large
  Speaker Recognition\\
  \normalsize
  Deep Learning for Audio Signals\\
\end{minipage}
\begin{minipage}{0.295\textwidth}
  \raggedleft
  \today\hfill\\
\end{minipage}
\medskip\hrule

%-------------------------------
%	CONTENTS
%-------------------------------

\tableofcontents

\section*{Introduction}

Speaker recognition is the task of identifying a person from his/her voice. In this lab report, we will examine a speaker recognition system that will use \emph{x-vectors} embeddings to classify the speakers. The system has been developed to solve the \href{https://www.robots.ox.ac.uk/~vgg/data/voxceleb/competition2020.html}{VoxCeleb} challenge.\\

The system \emph{extracts} the fixed-size embeddings trying to include in these representations as much discriminative information as possible for the speaker recognition task. A few architectures of neural networks can be used with this purpose, as well as different loss functions.\\

The code that I used to test the model, which was provided by \href{http://audias.ii.uam.es/staff/}{Alicia Lozano}, can be found in this \href{https://drive.google.com/file/d/1aWhFcbRUTJ0sOtYNPvLjHencfINu8SFz/view?usp=sharing}{link}.

\section{Examining the code}

In this section, we will examine the provided code in order to find the most relevant parts of it. Let us begin by examining the \inline{trainSpeakerNet.py} file, since it is the main script. The code is properly commented so identifying the different parts is quite simple.\\

\textbf{Questions.}\\

\begin{itemize}
  \item \emph{Which line of the code trains the model?}\\

        We found \inline{line 169}:
        \begin{minted}{Python}
    loss, traineer = s.train_network(loader=trainLoader);
  \end{minted}
        which trains the code. The variable \inline{s} was previously initialized as \inline{s = SpeakerNet(**vars(args));}, creating an instance of the SpeakerNet defined model class. This class has the called \inline{train_network} function, which performs (vaguely explained) a classic forward pass with loss computation and returns the average loss and the \inline{train eer}.

  \item \emph{Which command loads a pre-trained model?}\\

        Having a look from lines \(112-121\), we find that in lines \(116\) and \(120\), the function \inline{s.loadParameters(args)} is called. This function loads the pretrained model from a specific file. As we can see in the code, a distinction is made in our case, being able to load the model from the \inline{save_path} or from the argument \inline{args.initial_model}.
  \item \emph{Which line evaluates the performance of the neural network?}\\

        A few lines below the training line, in line \inline{176} we find:
        \begin{minted}[]{Python}
    sc, lab, _ = s.evaluateFromList(args.test_list, print_interval=100, test_path=args.test_path, eval_frames=args.eval_frames)
  \end{minted}
        which evaluates the model in the \inline{test_list}. In this function, all the features for each of the test file audios (\emph{.wav}) files are extracted and then the scores are computed using \inline{pairwise_distance}.

  \item \emph{Which variable contains the scores of the trials list? In which file are those scores saved?}\\

        In line \inline{177} we observe that the scores are thresholded and stored in the variable \inline{result}. Then, they are saved in the \inline{scorefile} file. This is a variable that, in line \(150\) opened a file called \inline{result_save_path/scores.txt}, where \inline{result_save_path} came as an argument from the execution of the script.

  \item \emph{Which variable controls the number of epochs(iterations) that the model is trained?}\\

        We find in line \(198\) and \(201\) that the variable \inline{it} is modified and checks the \inline{max_epoch} indicated as an argument.


\end{itemize}


\end{document}